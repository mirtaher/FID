\input{ /Users/Mohsen/Desktop/HelperFile.tex}
\doublespacing
\begin{document}

\title{Risk Sharing within Family with Endogenous Marriage Dissolution}
\author{Mohsen Mirtaher}
\date{}
\maketitle

\section{Introduction}



Risk sharing against income shocks is cited as one of the gains of marriage in the literature \citep{Chiappori_Mazzocco_2014}. Risk sharing via family operates through two mechanisms. First, shocks to an agent's wage can be compensated by the change in (intensive or extensive margin of) the spouse's labor supply. For instance, when the husband is hit by a health shock and become disabled, the wife can start working (extensive margin) or increase her hours of working by switching from a part-time job to a full-time one (intensive margin). The response in the extensive margin is known in the literature as the "added worker effect" \citep{Hyslop_2001}. \citet{Juhn_Potter_2007} finds that this insurance mechanism has weakened by a recent increase in correlation of employment and hours of work among couples. Second, conditional on employment of both wife and husband, a partner's negative income shock can be absorbed by the other partner via income pooling, barring a perfect positive correlation across spouses' income shocks. In the face of permanent and transitory income shocks, \citet{Blundell_etal_2015_Con-Ineq} try to find the quantitative importance of these family insurance mechanisms in comparison with the self-insurance via saving and government provided insurances such as transfers, tax credits, and progressivity of taxes. They find that after a \$100 permanent shock to the husband's wage, his contemporaneous consumption only responses as much as \$35, therefore, \$65 of the shock is insured. After breaking down the share of different insurance mechanisms, they find that income pooling within family and wife's labor supply response accounts for \$30 and \$21 worth of insurance, respectively, which is a large share. To put in perspective, self-insurance via savings and government each provides \$ 7 worth of insurance. Therefore, risk sharing within the family is an important insurance mechanism against income volatilities. \\ 

Nonetheless, \citet{Blundell_etal_2015_Con-Ineq} finding of the outsize contribution of family in providing insurance 
may partly be due to their sample selection, because they focus only on stable marriages. What if the arrival of income shocks endogenously make couples seek separation? \citet{Weiss_Willis_1997} find that a negative income shock to the husband's income increases the hazard rate of divorce whereas a positive one decreases it. In other words, despite the fact that family is an effective risk sharing means, it can be unraveled in the face of shocks. This insurance mechanism, as \citet{Krueger_Perri_2006_RES} name it, is "endogenous" meaning that the structure of uncertainty can influence the amount of insurance. \citet{Krueger_Perri_2006_RES} analyze a theoretical framework in which agents agree on risk-sharing. However, they could renege on their obligations at the expense of excluding from risk-sharing in the future. Marriage is a perfect example of such a contract. The partners commit to sharing the risk while they are together, yet they can get a divorce if their marriage participation constraint binds. Marriage is a contract written by partially committed parties; they are committed up to the point that they are better off with marriage as opposed to separation. In fact, modeling marriage as a partially committed arrangement is gaining traction in the literature. \citep{Chiappori_Mazzocco_2014, Mazzocco_etal_2013, Voena_2015} \\ 

\citet{Lise_Yamada_2014} empirically test the hypothesis of partial commitment and find that they could not reject it. Consistent with partial commitment, they find that the extent of insurance within the family is limited; if realized income shocks are small, the family provide insurance, but large realized shocks cannot be insured unless the Pareto weight between couples is updated accordingly. Therefore, when realized shocks are small, marriage remains intact and couples enjoy the benefit of risk sharing within family, but large realized shocks could dissolve the marriage and halts the risk sharing exactly at the time that there is the most need for insurance \footnote{ The limitation of the endogenous insurance could be mitigated by allowing the partners to rebargain about the Pareto weight before going straight for divorce. For instance, if the participation is binding for the wife, it is possible to save the marriage by updating the Pareto weight more favorably toward the wife so that she achieves at least the value of being single. This way, the husband is still better off relative to getting the divorce. In other words, if the participation constraint for the wife is binding with or without renegotiating she will get the value of being single, but allowing renegotiation benefits the husband.}.\\ 

In this paper, we study the risk-sharing within the family with limited commitment. We extend \citet{Blundell_etal_2015_Con-Ineq} by endogenizing the marriage dissolution while smoothing consumption in the face of income shocks. We examine how the family's contribution to providing insurance will be affected if we allow that marriage dissolution happens endogenously. One can expect that large share of the family in providing insurance, as is reported in \citet{Blundell_etal_2015_Con-Ineq}, falls and consumption inequality rises if the possibility of marriage dissolution is introduced to the analysis. Therefore, the first contribution of the paper is to measure how much the insurance mechanism provided by family is weakened if divorce is allowed to happen. \\ 

The endogenous nature of the insurance provided within the family can have important policy implications. In fact, the insurances provided by government can interact with insurance mechanisms embedded in the marriage. In particular, we study the progressivity of labor income taxes as an insurance offered by the government. Mechanically, progressive taxing is an equalizing policy. However, in addition, it can reinforce family insurance mechanisms by ameliorating the problem of limited commitment. Net of income effects, a progressive scheme relative to a comparable flat tax rate, makes it less likely for participation constraint to be binding because the discrepancy in after-tax wage shocks between couples is muted. In other words, everything else equal, a more progressive labor income tax scheme \footnote{After compensation to offset income effects that make the marriage participation constraints tighter.} reduces the divorce rate. But, on the other hand, if it applies in the long run, will reduce the variance income shocks; consequently it can substitute for marriage as a form of insurance. Therefore, it demands an empirical study to determine which one these opposing effects dominate.  \\

Before getting into consumption, we first need to scrutinize the link between income shocks and the divorce. How the features of spouses' income shock process affect the divorce rate? In particular, we are interested in the effect of persistence and variance of labor income shocks on the divorce. To this end, I write a two-period model. With uncertainty about income stream and persistence in income shocks. A married couple in the first period, after learning about one another's income shock, decide about divorce at the end of the first period. Because of the persistence of income shock, the realized first period shocks reveal some information about the status of the stochastic part of income at the second period. The model predicts that because of persistence the spouse with a better realized income shock in the first period has more incentive to end the marriage. This effect is stronger, the more persistent the income process is. In addition, we show that the spouse's willingness to divorce is higher when her income is less volatile relative to her husband and income shocks, at the second period, are less correlated between couples. In other words, the insurance provided by family is valued less for the spouse who enjoys less volatile income stream. In addition, the less correlated income shocks, the more the family can provide insurance. Next, we empirically test the predictions of the theoretical model. 

The rest of this paper is organized as follows. Section 2 lays out a simple theoretical model to demonstrate the way that uncertainty about income stream can entail to marriage dissolution. In particular, we study the effect of volatility and persistence in the income process on the likelihood of divorce. Section 3, provides some reduced form evidence for the effects of parameters of income process and probability of divorce by testing the predictions of the theoretical model. Section 4 developes a structural model.  \\   
 
 
\section{Two-Period Model} 

Following \citet{Hess_2004}, I write a two-period model to delineate how income shocks can lead to dissolution of a marriage. There are two periods in the model. At the beginning of the first period, everybody is married. After realization of the spouses' income shocks they decide about their consumption and saving. At the end of the first period, given the realized income shocks, couples decide to stay married or divorce at the second period. Finally, after realization of income shocks, each group of the married and divorced choose their consumption in the second period. 

\subsection{Environment}
The income process is modeled as the following:

\begin{align*}
y_{i,t} &= \bar y_{i,t} + \varepsilon_{i,t} \quad i \in \{h,w\} \quad t \in\{1,2\} \\
\varepsilon_{i,2} & = \mu \varepsilon_{i, 1} + \eta_{i} \quad \sigma^2_{\eta,i}\neq \sigma^2_{\eta,j}\\
\end{align*}


\noindent where $\bar y_{i,t}$ denotes the deterministic part of income.  $\mu$ captures the persistence of income shocks. Note that, in spite of persistence, the variance of the shocks is not the same between spouses. To focus on income shocks, we assume that the deterministic part of income follows the same scheme for the couple. In particular, spouses enjoy the same lifetime earning and get the share $\theta$ of that in the first period.

\begin{align*}
\bar y_{i1} & = \theta \bar y  \quad i \in \{h,w\}\\
\bar y_{i2} & =  (1+r) (1 - \theta) \bar y = \beta^{-1} (1- \theta) \bar y
\end{align*}

\noindent where the last equality come from the assumption that \((1+r) = \beta^{-1}\). To keep the closed-form solutions analytically tractable I make further simplifying assumptions. First, I assume a social welfare function while in a marriage which is additive separable in the couple's utilities with equal Pareto weights. The functional form of individuals' utility is specified as quadratic. 

\begin{equation*}
U(c_h, c_w) = \dfrac{1}{2}u(c_h) + \dfrac{1}{2}u(c_w)
\end{equation*}

\begin{equation*}
u(c) = c - \dfrac{b}{2} c^2
\end{equation*}

Finally, upon divorce, we assume that  the saving is divided equally, which is a realistic assumption given the law in most of the states in the US. 

\subsection{Solving the model}
Using the backward induction technique we solve the model as follows.

\subsubsection{Second Period}
The married couples jointly decide about their optimal consumption by solving the following problem 

\begin{align}
\max_{c_{h2}^M, c_{w2}^M} & \quad U(c^M_{h2}, c_{w2}^M) \nonumber \\
s.t. \sum_{k \in \{h,w\}} c_{k2}^M & = \beta^{-1} \left (S + 2(1-\theta) \bar y \right ) + \sum_k \varepsilon_{k2} \nonumber
\end{align}

\noindent where $S$ denotes the household's saving in the first period. FOCs for this problem readily imply

\begin{equation}
c_{h2}^M = c_{w2}^M \equiv c_{2}^M = \beta^{-1} \left ( \underbrace{\dfrac{1}{2} S+  (1-\theta) \bar y}_{B}\right ) + \dfrac{1}{2}\sum_k \varepsilon_{k2} 
\label{c2m}
\end{equation}

Upon divorce, we need an assumption about the way that physical assets are going to split. I assume that couples split  equally the saving.  Since, the agent would die at the end of this period, optimization problem for divorce people is trivial and they consume all of their resources. 

\begin{equation}
c_{i2}^D = \beta^{-1} \left ( \underbrace{\dfrac{1}{2}  S +  (1-\theta) \bar y}_{B} \right ) + \varepsilon_{i2} 
\label{c2d}
\end{equation}

Equations \eqref{c2d} and \eqref{c2m} entail that consumption in the second period is comprised of three components. First, the deterministic part of income in the second period. Second, the intertemporal transfer of first period income -both stochastic and deterministic parts- via saving. Third, the stochastic part of income realized in the second period. Comparing \eqref{c2d} and \eqref{c2m}  reveals that - with the aforementioned rule of resource division upon divorce - the only difference between married and divorced people in the second period is that divorcees consume their own income shock whereas the married individuals have access to an extra insurance mechanism thorough which they share their second period income shock. In other words, in the second period, the married have access to another form of insurance besides savings agains income shocks. 


\subsubsection{First Period}

In the first period, couples jointly decide about their private consumption and saving.  

\begin{align*}
\max_{c_{h1}^M, c_{w1}^M, S} \quad  &U(c_{h1}^M, c_{w1}^M) + \beta \left \{ D E_1[ U(c^D_{h2}, c^D_{w2})] + (1-D) E_1 [U(c_{2}^M, c_{2}^M)] \right \}  \\
s.t. &\sum_{k \in \{h,w\}} c^M_{k1} + S =  2 \theta  \bar y+ \sum_k \varepsilon_{k1}  \label{p1constraint}
\end{align*}


\noindent solving the problem implies the following Euler equation 


\begin{equation}
c^M_{h1} = c^M_{w1} \equiv c_1^M = D \left( \dfrac{1}{2}E_1[c^D_{w2}]  + \dfrac{1}{2}E_1 [c^D_{h2}] \right) + (1-D) E_1[c_2^M]  \label{cd}
\end{equation}

\noindent where $E_1$ denotes the expectation conditional on revealed information by the first period.  Equation \eqref{cd} can be explained intuitively. Remember that, in the single-agent problem with quadratic preferences, consumption smoothing implies that the first period consumption is equal to the expected second period consumption. Equation \eqref{cd} is the extension of that result. The couple will equate their first period consumption with the weighed average of their expected consumption in the state of divorce if the marriage is dissolved and with their expected consumption in the state of marriage if the marriage is remained intact. By substituting \eqref{c2m} and \eqref{c2d}  in \eqref {div} and taking expectation

\begin{equation}
c^M_{1} = \dfrac{ \mu}{2}   \left ( \varepsilon_{h1} + \varepsilon_{w1} \right )+ \dfrac{\beta^{-1}}{2} \left(  S + 2(1-\theta) \bar y\right)
\label{c1mh}
\end{equation}

\noindent By substituting \eqref{c1mh} and \eqref{c1mw} in the first period constraint \eqref{p1constraint}


\begin {align}
(1 + \beta^{-1}) S  &=  2 \left (\theta - \beta^{-1} (1- \theta)  \right ) \bar y + \left ( 1 - \mu \right ) \sum_k \varepsilon_{k1} \\
c^M_1 & = \dfrac{\bar y}{1+\beta} + \left( \dfrac{1-\mu}{1+\beta} + \mu \right) \dfrac{\sum \limits _{k} \varepsilon_{k1}}{2}
\label {saving}
\end{align}


Note that the saving is independent of the decision to divorce, because the social welfare function weighs spouses equally just as the saving sharing if divorce happens. Thus, the optimization problem in the first period is solved by the following order. First, the couple find the optimal level of joint saving according to \eqref{saving}. Second, the decision about divorce is made. Third, they choose the optimal consumption levels in the first period. \\

\begin{comment}
In fact, the order is as follows. In the first period they come up with optimal consumption and saving contingent on D. Then the husband using these contingents make decision and then after decision consumption-saving are pin down.
\end{comment}

Equation \eqref{saving} shows that two forces in the model leads to saving. First, if there is skewness in arrival of deterministic income across two periods, the couple use the saving technology to smooth consumption. Second, the couple transfer part of the realized income shock at the first period to the second period. However, this transfer depends on the degree of persistence in income shocks. The more persistent are the shocks, the less incentives the couple have to save a part of their first period shock, because they anticipate to hit with more or less similar shock at the second period. In the extreme case of random walk income process ($\mu = 1$) in which any shock in the first period fully carries to the second period, the couple fully consume their first period shock and do not save any part of it. The discussion is summarized in the following proposition.

\begin{prop}
 The joint saving is a decreasing function of persistence in income shock process ($\mu$). 
\end{prop}
 

\subsection{Decision to Divorce}
The decision to divorce is made unilaterally. Without loss of generality, assume that the husband is the one who is receiving a better shock in the first period, consequently he has higher bargaining power and get to decide about the fate of the marriage. At the end of the first period and after realization of first-period income shocks, the husband finds it optimal to walk away from marriage if his expected second-period utility under divorce is higher than that under marriage:

\begin{equation}
D_h = \mathbf 1 \left \{ E_1 u(c_{h2}^D) > E_1 u(c_2^M) \right \}
\label{div}
\end{equation}

\noindent By substituting \eqref{c2m} and \eqref{c2d}  in \eqref {div} and taking expectation conditional on revealed information at the first period:

\begin{align*}
D_h = \mathbf 1 \left \{u(\beta^{-1}B + \mu \varepsilon_{h1}) - u(\beta^{-1}B + \mu \dfrac{\varepsilon_{h1} + \varepsilon_{w1}}{2}) > \dfrac{b  (1- \Omega )\sigma^2_{h,\eta}}{2} \right \}
\end{align*}

Or equivalently, 

\begin{align}
D_h = \mathbf  1  \left  \{ \underbrace{  \mu (\varepsilon_{h1} - \dfrac{\varepsilon_{h1} + \varepsilon_{w1}}{2} ) - \dfrac{b}{2} \mu^2 (\varepsilon^2_{h1} - \dfrac{(\varepsilon_{h1} + \varepsilon_{w1})^2}{4})}_{u(\mu \varepsilon_{h1}) - u(\mu \frac{\varepsilon_{h1} + \varepsilon_{w1}}{2})} \right. \nonumber \\
\left.   - \underbrace{  \dfrac{b }{4(1+\beta)}  \mu (\varepsilon_{h1} - \varepsilon_{w1})   \sum_k \bar y_k }_{\text{income effect due to deterministic part of income}}\right. \nonumber \\
\left.   -  \underbrace{\dfrac{b }{4(1+\beta)} \mu (\varepsilon_{h1} - \varepsilon_{w1}) (1-\mu) (\varepsilon_{h1} + \varepsilon_{w1})}_{\text{income effect due to saving}}  \right.  \nonumber \\ 
 \left.     >  \underbrace{\dfrac{b  (1- \Omega )\sigma^2_{h,\eta}}{2}}_{\text{threshold}}   \right \}
\end{align}

 where  $\Omega = \dfrac{1 + 2\rho_\eta \Phi+ \Phi^2}{4}$, $\Phi = \dfrac{\sigma^2_{w,\eta}}{\sigma^2_{h,\eta}}$. %$B = \dfrac{1}{2} \sum_k s_k + \dfrac{1}{2} (1-\theta) \sum_k \bar y_k$. 

The LHS of the inequality is consisted of three terms. In order to interpret them, remember that consumption at the second period is financed by three resources: second period income shocks, deterministic part of income, and saving. In addition, among three resources, only income shocks are affected by decision to divorce and saving and deterministic income is the same whether married or divorce. Thus the difference in consumption between divorcees and the married is due to second period income shocks: the married share equally their income shocks whereas divorcees fully consume their own income shock. \\

To interpret different components of LHS, since we assumed that the husband receives a better shock compared to his wife at the first period, $ (\varepsilon_{h1} - \varepsilon_{w1}) > 0$.  Thus, strictly increasing property of the utility function, $u(.)$, implies that the first terms is positive. This result is intuitive because if husband receives a relatively better shock at the first period, given the persistence of income process, he also expects the upper hand at the second period as well. Therefore, for the husband, the utility of expected income shock while divorced is higher than the utility of expected income shock while married. The first term captures the difference in utility when theres is no saving (and deterministic income), and income shocks are the sole source of consumption. \\

The second and third terms appear because the consumption also is financed by deterministic part of income and saving, which are the same across marriage and divorce. The second term is negative to correct for the income effect emanating from deterministic income. The third term has ambiguous sign and accounts for saving. To gain intuition, when saving is positive, the marginal utility of consumption at the second period is higher compared to the case that saving is zero (i.e. the first term); therefore the first term overstates the extra utility gained by getting divorce. In contrast, when saving is negative (i.e. borrowing), the marginal utility of consumption at the second period is higher compared to the case that saving is zero (i.e. the first term); therefore the first term understates the extra utility gained by getting divorce. Thus, the third term is positive when saving is negative and vice-versa. A parallel argument justifies why the second terms is appeared with negative sign. The existence of deterministic income as another source of consumption lowers the marginal utility of consumption, consequently, it makes the extra gain of getting divorce calculated in the first term exaggerated. The second term shows up negatively to correct for it.  \\

By now, it should be clear that concavity of utility function plays a big role in decision to divorce characterized in  \eqref{div}. In fact, if the agents were risk-neutral and utility function was linear, the divorce decision would be characterized simply: the husband will get divorce if he has a relatively better shock at the first period. \\

\begin{equation*}
D_h = \mathbf 1 \left \{ \dfrac{\mu}{2} (\varepsilon_{h1} - \varepsilon_{w1}) > 0\right\}
\end{equation*}

Next, we investigate how the likelihood of a divorce is affected by persistence of the income shock process, $\mu$, given the realized first period shocks? The short answer is that it is theoretically ambiguous. For the rest of our discussion keep the assumption that it is the husband who enjoys the better income shock at the first period. An increase in $\mu$, reinforces the better position of the husband's income at the second period and widens the income gap between the state of divorce and marriage. However, higher income in the state of divorce means lower marginal utility of consumption. Thus, in the state of divorce relative to marriage, each unit of extra income generates less utility even though the income gain is higher. Therefore, it is ambiguous whether the marginal utility gain in the state of divorce beats that of marriage following an increase in persistence of income shocks. In addition, the reaction of saving even further complicates the problem. Assume that the sum of couples' shock at the first period has been positive so that they could save part of it. We discussed that as the persistence of income shock goes up a smaller fraction of first period shocks are saved. In the extreme case of random walk, none of the first period shocks are saved. Thus, with higher persistence, the dampening effect of (positive) saving to correct for marginal utility lowers. On the other hand, with lower persistence, the difference between divorce and staying married in terms of predictable income becomes smaller, thus, the correcting effect of saving is smaller, too. Therefore, the corrective effect of saving disappears both for highly persistent and non-persistent income processes. In fact, in this model, it is maximized when $\mu = 0.5$. The above discussion is summarized in the following proposition:

\begin{prop}
The effect of an increase in persistence of income process, $\mu$, on willingness to divorce, is ambiguous because of confounding income effects. However, if we assume that the price effect is dominant, it will increase the likelihood of a divorce. 
\end{prop}

 
What is the role of uncertainty in decision to divorce in \eqref{div}? Uncertainty about income at the second period creates the threshold in the RHS of the equation \eqref{div}. With no uncertainty, this threshold would be zero, thus, regardless of how concave the utility function is, the spouse with the better shock at the first period will seek divorce because she/he  will enjoy certainly higher income at the second period as well due to persistence in income process.  However, with the introduction of an income shock in the second period, a risk-averse agent, will take into account the distribution of joint income shock. The following proposition formalizes how properties of joint income shock distribution affect the decision to divorce. 

\begin{prop}
Without loss of generality, assume that the husband is hit by a better income shock in the first period. Then, one sufficient condition for divorce is that the husband's variance of shocks is small enough relative tho that of his wife, i.e.,  \( \dfrac{\sigma_{h, \eta}}{\sigma_{w,\eta} } < d^*(\rho_\eta)\). Furthermore, the threshold is increasing in correlation of shocks between husband and wife,  \( \dfrac{\partial d^*}{\partial \rho_{\eta}} > 0\), implying that the husband's willingness to divorce is weakly increasing with respect to the correlation of shocks, $\rho_\eta$.  \label{prop.2}
\end{prop}

The intuition behind the proposition \eqref{prop.2} is clear. Provided that the husband has been benefited a better income shock at the first period, due to persistence in income process, he will also expect a better shock in the second period, unless the the variance of his transitory shock is high. In fact, so long as the variance of his transitory shock is small relative to the wife's,  \( \dfrac{\sigma_{h, \eta}}{\sigma_{w,\eta} } < d^*(\rho_\eta)\), unambiguously he would seek divorce. However, if the husband has a relatively more dispersed shocks, he highly values the risk sharing function of the marriage. Furthermore,  as the correlation of shocks between spouses increases, the insurance capability of the marriage decreases. As a result, even with a higher variance, the husband may still be better off with divorce. In other words, with higher correlation of shocks, the ability of a marriage in providing insurance against income shocks weakens.  \\

In this theoretical model, the divorced agent can only live as a single in the second period. However, in reality, when deciding about divorce, people take into account the option value of remarriage following getting divorced. Adding this mechanism to the model may increase the value of divorce and push people even further toward divorce. We incorporate this mechanism into the structural model and study it more closely in section \ref{remarriage}. 

\subsection{Collective Model}

In the last section, we did the theoretical analysis using a unitary framework, in that, the arrival of shocks did not change the intra-household allocation parameter. Collective model is a more general framework allowing the allocation parameter to respond to shocks. Figure \ref{intra} illustrates how the intra-household allocation margin can affect the the insurance capacity of the family. In the figure, the husband is hit by a negative shock where shifts in the Pareto frontier and reduces the husband's reserve outside option from D to F. In the case of full commitment, that is, not seeking divorce and committing to the initial Pareto weight, $\mu_0$ utility loss is as much as AB. Permitting limited commitment, the wife no longer can benefit from this marriage as het reserve utility is more than what she gets in the marriage. When divorce happens the husband's utility drops from A to F. However, if we let the couple re-bargain about the Pareto weight, they can save the marriage by updating that in favor of the wife. The red ray shows the the minimum Pareto weight that makes the wife indifferent  between staying in the marriage and collecting her outside option. By this minimum accommodation, the husband's utility falls from A to C, thus the amount of loss is greater as much as BC relative to full commitment. However, It is not clear why the wife should be satisfied by this bare minimum. In fact, the husband can end up anywhere between B and F depending on the how to model the process of bargaining. Therefore, by using the collective model the solution would be indeterminate without specification of the bargaining process, however it is common in the literature to stick with the bare minimum updating \citep{Chiappori_Mazzocco_2014}. \\


\begin{figure}
\centering
\begin{subfigure}{0.65\columnwidth}
\includegraphics[width= \textwidth]{/Users/Mohsen/Dropbox/Thesis/intra_hh_updating}
%\caption{}
\label{fig: intra}
\end{subfigure}
\qquad
\caption{Intra-household allocateion response and insurance against shocks}
\label{intra_hh}
\floatfoot{The husband is hit by a negative shock where lowers his outside options and shifts in the pareto frontier without changing the outside option if the wife. After shock, the wife would be better off with divorce unless the intra-household allcation rule is updated in favor of her so that she at least achieves her outside option. The figure compare three different scenarios regarding commitment within marriage in the face of negatie shocks: Full commitment, limited commitment without rebargaining, and limited commitment with bargaining. }
\end{figure}



\section{Reduced Form Evidence}

In this section, we test the predictions of the theoretical model reviewed as follows. We assume that the households are heterogenous in terms of their income process parameters, in that, the spouses' variance, covariance, and persistence of the shocks vary across households, but they are stationary. The first prediction is that a marriage is more likely to dissolve the bigger is the discrepancy between husband's and wife's variance of the shocks. Second, the ability of a family in providing insurances diminishes as the correlation of shocks between husband and wife grows. Third, given that the income effect is dominated, a higher persistent income process causes a higher probability of divorce. \\

The first step is to identify the parameters of the income shock process. Recall that income shocks follow AR(1) process:

\begin{align*}
y_{i,t} &= \bar y_{i,t} + \varepsilon_{i,t} \quad  i \in \{H,W\} \\
\varepsilon_{i,t} & = \mu_i \varepsilon_{i, t-1} + \eta_{i} \quad \sigma^2_{\eta,H}\neq \sigma^2_{\eta,W}\\
\end{align*}

\noindent So, the income process parameters are identified as follows: 

\begin{align*}
\mu_i &= \dfrac{Cov(y_{i,t}, y_{i, t-1})}{Var(y_{i,t})}  \\
Var(\eta_{i,t}) &= (1 - \mu^2) Var(y_{i,t})\\
Cov(\eta_{H,t},\eta_{W,t}) &= (1 - \mu^2 ) Cov(y_{H,t}, y_{W,t}) \\
\end{align*}

\noindent since the identification is constructive, assuming stationarity, we can estimate the parameters using within household estimators:

\begin{align*}
\hat \mu_i &= \dfrac{\sum \limits_{\tau = 2}^{T} (y_{i,\tau} - \bar y_i) (y_{i, \tau-1} - \bar y_i)}{\sum \limits_{\tau = 1}^{T} (y_{i,\tau} - \bar y_i)^2} \\
\widehat{Var(\eta_{i,t})} &= \dfrac{(1 - \hat \mu_i^2)}{T} \sum \limits _{\tau = 1}^T y^2_{i,\tau}\\
\widehat {Cov(\eta_{H,t},\eta_{W,t})} &= \dfrac{(1 - \hat \mu_W)(1 - \hat \mu_H)}{T} \sum \limits_{\tau =1}^T (y_{H,\tau} - \bar y_H) (y_{W,\tau} - \bar y_W)\\
\end{align*}

Next, we estimate the effect of the income process parameters on hazard rate of divorce via a proportional hazard and logic model. In particular, the hazard rate of divorce is multiplicative separable with respect to time as follows. 

\begin{align*}
\lambda(t, X_m) &= \lambda_0(t) \exp(\boldsymbol \beta' X_m) \label{hazard}\\
\boldsymbol \beta' X_m &= \beta_1 \widehat {Cov(\eta_{H,t},\eta_{W,t})} + \beta_2 \hat \Phi + \beta_3 \hat \mu_H + \beta_4 \hat \mu_W + \boldsymbol \beta_5 W
\end{align*}

\noindent where $\lambda(t, X_m)$ is the hazard rate of divorce as a function of time and time-invariant covariates, $X_m$, including the features of income shocks and other controls, $W$. Controls are race, education, age at the time of first marriage, the presence of kids before marriage, and the divorce status of parents as there are evidence for intergenerational transmission of divorce. $\hat \Phi$ is a measure the distance between the husband's and wife's variance of shocks 
$\hat \Phi = \max \{\dfrac{\widehat {Var(\eta_{H,t})}}{\widehat{ Var(\eta_{W,t})}}, \dfrac{\widehat {Var(\eta_{W,t})}}{\widehat{ Var(\eta_{H,t})}}\}$. \\

One potential problem with model \eqref{hazard} is that there could be some omitted variables where affect both the income process parameters and the likelihood of divorce such as depression. To address this problem, we use an instrumental variable that affect the likelihood of divorce only through affecting the process of income shocks. The instrument is the occupational parings of spouses. The occupation is clearly associated with the type of income shocks where workers in that occupation experience. In addition, we assume that, other than income channel, an occupation has no impact on divorce.  Thus, we run the following first stage regression in which the income process parameters estimates are regressed on occupational pairs, Z, as an instrument and other controls in equation \eqref{hazard}. 

\begin{equation*}
p = g_p(Z, W) +\nu_p \quad p \in \{Cov(\eta_{H,t},\eta_{W,t}), \hat \Phi, \hat \mu_H, \hat \mu_W \}
\end{equation*}

Then, we use $\hat g_p(Z, W)$ instead of endogenous covariates in \eqref{hazard}. \\



THE EMPIRICAL RESULTS WILL BE AVAILABLE SHORTLY...



  

\section{Structural Estimation}

Since the majority of people remarry after divorce we allow the agents to remarry once after divorce. However, since we treat marriage formation exogenously,  following \citet{Voena_2015}, we model remarriage as a random event with Bernoulli distribution. Therefore, we focus on the subsample that at most one divorce and one remarriage is happened. Thus, there are three possible marital states at each period $t$: married, divorced, and remarried; $s_t \in \mathcal S = \{M, D, R \}$. Remarriage, $R$,  is the absorbing marital state.  There are two types of individuals: husbands and wives $j \in \{ H, W\}$. Denote $t^D, t^R$ as the period at which the switch to divorce and remarriage happen, respectively. \\


\subsection{Utility Function}
The flow utility depends on consumption, $c_t^j$, labor supply $h_t^j$, and marriage. Labor supply is discrete but, to allow the intensive margin, part-time working is allowed. Thus, labor supply can take on three values, full time $h_t^j = 1$,  part time $h_t^j = \frac{1}{2}$, or not working $h_t^j = 0$. Since we do not model marriage formation, we allow for initial heterogeneity. In particular, we allow for heterogeneity in marginal utility of consumption, leisure, and marriage by considering $Q$ discrete  unobservable types of individuals. We also introduce stochastic shocks to  marginal utility of consumption, leisure, and marriage. To allow for risk averse agents, we introduce the concavity in utility function with respect to consumption. In addition we allow for complementarity between consumption and leisure. Finally, state dependence in labor supply from unemployment to full time or part time job is taken into account.  Utility function is specified as the following \footnote {To avoid the clutter of indices I drop the the index for household and gender. To be precise, $c^j_{i,t}$ denotes the consumption of the partner of gender $j$ in the household $i$ at period $t$. }:
  
\begin{align*}
 u(c_{t}, h_{t}, m_{t}; \ce{^{c}\epsilon_{t}}, \ce{^{h}\epsilon_t}, \ce{^{\mathbf{m}}\mathbf{\epsilon_t}}) = & \dfrac{c_t^{\sum \limits_ {i=1}^{Q} \alpha_iI(type = i)}}{{\sum \limits_ {i=1}^{Q} \alpha_iI(type = i)}} \left (1 + e^{\beta_1 h_t+\ce{^c\epsilon_t}} \right ) \\
& + h_t \left( \sum \limits_{i=1}^{Q} \left( \beta_{i+1} I(type =i) + \beta_{i+1,W}  I(type = i, gender = W)\right) \right. \\
& \left.  + \beta_{Q+2} I(race = Black) + \beta_{Q+3} I(race = Hispanic) +\ce{^h\epsilon_t}  \right) \\
& + \beta_{Q+4} I(h_t =\frac{1}{2}, h_{t-1} = 0) + \beta_{Q+5} I(h_t = 1, h_{t-1} = 0) \\
& + \sum \limits_{i=1}^{Q} \sum \limits_{l=1}^{Q} \left( \beta_{Q+5+i+l} + \ce{^{m,l}\epsilon_t}\right ) I(\text{Wife's Type} = i, \text{Husband's Type} = l) I(m_t = 1)
\end{align*}


where $\mathbf {\epsilon_t^m}$ is the vector of love shocks for different types of marriages. $m_t$ is a binary variables which is equal to 1 for the married. 

\subsection{Pareto Weights}

 We assume that Pareto weights are set at the beginning of the marriage and remains the same thought the marriage. In other words, we don not allow for renegotions about Pareto weight in the course of marriage. However, couples are not fully committed to their vows and marriage dissolves when the participation constraint of at least one of the spouses bind according to unilateral divorce laws. To ensure that the husband's Pareto weight \footnote {To avoid the clutter of indices I drop the index referring to the household index. To be precise, $\theta_i^H$ denotes the husband's Pareto weight in the household $i$.}$\theta^H \in (0,1)$, we choose a logistic functional form as the following: 

\begin{align*}
\theta^H = & \left ( 1 + \exp \left( \theta_1 (w^H_{0} - w^W_{0}) + \theta_2 \log \left( \dfrac{E^H}{E^W} \right) + \theta_3 \log \left ( \dfrac{K^H_{0}}{K^W_{0}}\right ) + \theta_4 sr +\theta_5 \log \left ( \dfrac{\ce{^p y^H}}{\ce{^p y^W}}\right ) )\right. \right. \\
& \left. \left. + \theta_6 (t^H - t^W) + \theta_7 (t^H - t^W)^2 +  \sum \limits_{i=1}^{Q} \sum \limits_{l=1}^{Q} \theta_{7+i+l} I(\text{Wife's Type} = i, \text{Husband's Type} = l)   \right )\right )^{-1}
\end{align*}

where \( e_j, K_{0j}, y_j^P, t_j\) are j's education, initial experience, parent's income, and age. $sr$ is the sex ratio where the couples meet. \\

\subsection{Wage Process}

To identify the wage shocks properly we need to control for observables that can predict wages. We control for part/full time jobs as they have systematic differences in terms of wage rate. We also include the conventional covariates in the Mincer equation, namely, education and a quadratic expression of experience, and demographics. In addition, we control for state of residence and the dummy for living in a large city. Also, we include the interaction terms of year with education, race, and state of residence.  
%Finally, we control for unobserved heterogeneity among individuals and their interaction with gender.%
 The wage variation not explained by these covariates is considered as wage shocks, which are not predictable by agents. The wage of individual of gender $j$ in the household $i$ at age $t$ is explained as follows: 

\begin{align*}
\log w_{i,t}^j = & \omega_1 K_{i,t}^j + \omega_2 (K_{i,t}^j)^2 + \omega_3 E_{i,t}^j + \omega_4 I(h_{i,t} ^j = \frac{1}{2}) +   \omega_5 I(race = Black) + \omega_6 I(race = Hispanic) \\
& + \sum \limits_{\tau =t_0}^{T}  \left ( \omega_{6,\tau} E_{i, \tau}^j + \omega_{7, \tau} I(race = Black) + \omega_{8, \tau} I(race = Hispanic) \right )I(year = \tau)  \\
& + \varepsilon^j_{i,t} \\
\end{align*}

where the human capital is accumulated according to the following rule:

\begin{equation*}
K_{i,t}^j = K_{i,t-1}^j + h_{i,t}^j
\end{equation*}


There are two approaches to model the wage shocks process: HIP and RIP. In the HIP approach, the income shocks are modeled by a deterministic heterogeneous growth rate, a persistent  component, and a non-persistent transitory competent. Thus the income shock is modeled as follows:

\begin{align*}
\varepsilon^j_{i,t} & = \xi^j_{i,t} + u^j_{i,t}\\
\xi^j_{i,t} &= \rho \xi^j_{i,t-1} + \vartheta^j_{i,t} \\
\omega_1 &=  \sum \limits_{i=1}^{Q} \left( \nu_{i} I(type =i) + \nu_{i,W}  I(type = i, gender = W)\right ) \\
\omega_2 &=  \sum \limits_{i=1}^{Q} \left( \nu_{i+Q} I(type =i) + \nu_{i+Q,W}  I(type = i, gender = W)\right )
\end{align*}

where $\xi_{i,j,t}$ is the persistent stationarity component and $u_{i,j,t}$ is the measurement error. The coefficients on experience, in the wage process is heterogeneous with respect to individual types and gender to capture the deterministic heterogenous part of income growth.  We assume that $\vartheta_{i,j,t}$ and measurement error $u_{i,j,t}$\ are not correlated across time and spouses.\\

In contrast, in the RIP approach there is no heterogeneity in income growth, but the wage shocks are compromised of two components: a random walk component and a transitory component. 

\begin{align*}
\varepsilon^j_{i,t} &= p^j_{i,t} + \zeta^j_{i,t} + u^j_{i,t} \\
p^j_{i,t} &= p^j_{i,t-1} + \eta^j_{i,t}
\end{align*} 

where $u_{i,j,t}$ is the measurement error, $p_{i,j,t}$ is the permanent component of income shock, and $\zeta^j_{i,t}$ is the transitory component of income shock. We make the following assumptions about the covariance structure of shocks:

\begin{equation*}
E(\zeta^j_{i,t} \zeta^k_{i,t-s}) =
 \begin{cases}
\sigma^2_{\zeta_j} & \mbox{if j = k, s=0} \\
\sigma_{\zeta_j \zeta_k} & \mbox{if j $\neq$ k, s=0} \\
0 & \mbox{otherwise}
\end{cases}
\end{equation*}

\begin{equation*}
E(\eta^j_{i,t} \eta^k_{i,t-s}) = 
 \begin{cases}
\sigma^2_{\eta_j} & \mbox{if j = k, s=0} \\
\sigma_{\eta_j \eta_k} & \mbox{if j $\neq$ k, s=0} \\
0 & \mbox{otherwise}
\end{cases}
\end{equation*}

\begin{equation*}
E(\zeta^j_{i,t} \eta^k_{i,t-s}) = 0 \quad \text{for all $j,k = {H, W}$ and all $s$}
\end{equation*}

\subsection{Fertility}

Fertility and having kids can change the labor supply decision of women, therefore ignoring fertility and treating it exogenously can conflate fertility with a negative shock to the wife's income. However, in order to reduce the computation burden and avoiding introducing a new state variable following \citet{Klaau_1996}, we assume that fertility follows a stochastic process. Since fertility is not the main focus of the paper, by modeling it as a stochastic process  instead of an endogenous deacons, we trade a new choice variable with more parameters (parameters of the process). we define the binary variable $\kappa_{i,t} =1$ when there is a child in the household $i$ at period $t$ (regardless their number and their age). we assume that $\kappa_{i,t}$ moves according to the following stochastic process: 

\begin{align*}
Pr(n_t =1 | n_{t-1} =0) &= \Phi(^fX'_t \mathbf{\beta_f}) \\
Pr(n_t = 1 | n_{t-1} =1) &= 1 \\
Pr(n_t = 0 | n_{t-1} = i ) &= 1 - Pr(n_t =1 | n_{t-1} = i) \quad i = 0,1
\end{align*}

\noindent where $^fX_t$ represents female's marital status in the previous period, $m_{t-1}$, education, race, and linear and quadratic of age. $\mathbf {\beta_f}$ represents the parameters should be estimated in fertility stochastic process. 
\footnote{A more accurate approach is to model a stochastic pregnancies which would be a function of the number of kids in different ages. To do so, we create stat e variables store the number of children in different age groups. We include these state variables  and their quadratic terms in the stochastic process. }


\subsection{Types and Initial Conditions}

In out model we treat education, the experience before marriage, and marriage formation as exogenous. However, if there were correlation between these variables and the structure of wage shocks and the insurance mechanisms in the model, our estimation of the insurance value of family would be biased. To address this problem, we assume that there are $Q$ discrete individual types that the initial condition variables can be considered exogenous after conditioning on the types. \\

To account for initial conditions and to control for unobserved heterogeneity, we assume that there are $Q$ discrete types of individuals. Assuming that $Q =2$, the type probability is given by the following function: 

\begin{multline*}
Pr(type_{i}^j = 2)  = \gamma_0 + \gamma_1 w_{i,0}^j + \gamma_2 K_{i,0}^j + \gamma_3 I(h_{i,0}^j = 0) + \gamma_4 I(h_{i,0}^j = 1) + \gamma_5 I(j = W) \\
+  \gamma_6 I(race_i^j = Black ) + \gamma_7 I(race_i^j = Hispanic ) + \gamma_8 E_{i,0}^j  + \gamma_9 t_{i,0}^j 
\end{multline*}

\subsection{The Remarried Problem} \label{remarriage}

For tractability, we assume that the Pareto weight for the remarried, $\lambda$, is determined exogenously and is independent of individual specific characteristics of spouses as in reality it is hard to predict the qualities of the matching partner in the second marriage. Also we assume that remarriage is the absorbing state in which the marital state cannot change. In fact the couple need to make decision about consumption, saving, and labor supply, $q_t = \{c_t^H, c_t^W, h_t^H, h_t^W,A_{t+1}\}$. Thus, the couple solve the following problem: 

\begin{align}
\ce{^{R}V_t}(\Omega_t)  &= \max_{q_t} \quad  \lambda u(c_t^H, h_t^H, m_t =1) + (1 - \lambda) u(c_t^W, h_t^W, m_t = 1) + \beta E(^RV_{t+1} (\Omega_{t+1})| \Omega_t, q_t) \nonumber \\
s.t. & \quad  \underbrace{\left ((c_t^H )^\phi + (c_t^W )^\phi  \right )^{\frac{1}{\phi}} e(n_t)}_{x_t} + A_{t+1} = (1+r) A_t + w_t^H h_t^H (1- \tau) + w_t^W h_t^W (1 - \tau) \nonumber 
\end{align}

denoting the solution for the above optimization problem by $\hat q_t$, the value of being remarried for a partner of gender $j$  given the next period value is:

\begin{equation*}
^{R}V_t^j (\Omega_t) = u(^0 \hat c_t^j, ^0 \hat h_t^j, m_t = 1) + \beta E \left (^RV^j_{t+1}(\Omega_{t+1}) | \Omega_t, ^{0} \hat q_t \right ) 
\end{equation*}





\subsection{The Divorcee Problem}
A divorce person makes decision about her consumption, labor supply, and saving:

\begin{equation*}
 ^{D}q_t^j = \{c_t^j, h_t^j, A_{t+1}^j \} \quad  j \in \{H, W\}
\end{equation*}

\noindent The state variables for a divorced person are 

\begin{equation*}
\Omega_t = \{ A_t^j, p_t^j, K_t^j \} 
\end{equation*}

\noindent In each period t, a divorced person can remarry with probability $\pi_t^j$  determined exogenously by age and gender. Thus, a divorced person solve the following problem: 

\begin{align*}
^{D}V_t^j (\Omega_t) &= \max_{^{D}q_t^j} u(c_t^j, h_t^j, m_t = 0) + \beta \left \{ \underbrace{\pi_t^j  E \left (^{R}V_{t+1}^j (\Omega_{t+1}) | \Omega_t, ^{D}q_t^j  \right )+ (1-\pi_t^j ) E \left (^{D}V_{t+1}^j (\Omega_{t+1}) | \Omega_t, ^{D}q_t^j  \right )}_{^{RD}V^j_{t+1} (\Omega_t)}\right \} \\
& s.t. \quad  c_t^j e(n_t) + A_{t+1}^j = (1+r) A_t^j + w_t^j h_t^j (1 - \tau)
\end{align*}

given next period value functions, $^{R}V_{t+1}^j (\Omega_{t+1})$ and $^{D}V_{t+1}^j (\Omega_{t+1})$, the above problem is a static optimization problem. Denoting the optimum solution by $^{D}\tilde q_t^j$, the value of being single at period $t$ for a person of gender $j$ is

\begin{equation*}
^{D}V_t^j (\Omega_t) = u(\tilde c_t^j (\Omega_t), \tilde h_t^j(\Omega_t), m_t = 0) + \beta \enskip ^{RD}V^j_{t+1} (\Omega_t)
\end{equation*}


\subsection{The Married Problem}

The couples in each period should decide about each spouse's consumption, their labor supply, and the family saving $q_t = \{c_t^H, c_t^W, h_t^H, h_t^W,A_{t+1}\}$ in addition to decision about continuing the marriage or getting divorce, $d_t$. The state variables for the household \footnote{ To avoid clutter in indices, I drop the household index.} decision is 

\begin{equation*}
\Omega_t = \{ A_t^H, A_t^W, K_t^W, p_t^H, p_t^W  \} 
\end{equation*}

The value of marriage is determined as the following 

\begin{equation*}
^{M}V_t = \max_{d_t \in \{0,1\}} \{ \ce{^{M0}V_t}, \ce{^{M1}V_t} \}
\end{equation*}

where $d_t = 1$ denotes the decision to divorce. To define the value function assume that the leading period value function, $V_{t+1}$ are known. 

\begin{align}
\ce{^{M0}V_t}(\Omega_t)  &= \max_{q_t} \quad  \theta^H u(c_t^H, h_t^H, m_t =1) + (1 - \theta^H) u(c_t^W, h_t^W, m_t = 1) + \beta E(^MV_{t+1} (\Omega_{t+1})| \Omega_t, q_t) \nonumber \\
s.t. & \quad  \underbrace{\left ((c_t^H )^\phi + (c_t^W )^\phi  \right )^{\frac{1}{\phi}} e(n_t)}_{x_t} + A_{t+1} = (1+r) A_t + w_t^H h_t^H (1- \tau) + w_t^W h_t^W (1 - \tau) \nonumber \\
& \underbrace{u(c_t^j, h_t^j, m_t = 1) + \beta E \left (^MV^j_{t+1}(\Omega_{t+1}) | \Omega_t, q_t \right )}_{^{M}V_t^j (\Omega_t)}  \geq \enskip  ^{D}V_{t}^j (\Omega_t)\quad j \in \{ H, D\} \label{participation}
\end{align}


where $E$ is taken with respect to utility and wage process shocks.  $e(n_t)$ is an equivalence scale that take into account the number of children $n_t$ on inflation of the couples consumption. $x_t$ captures the economies of scale that couples enjoy in their consumption technology by forming a family with $\phi \geq 1$.  \\

Given the next period value functions, $^MV_{t+1}, ^MV^j_{t+1}$, the above problem becomes a static optimization problem with two inequality constraints representing the participation constraints \eqref{participation} for each of the spouses. Denoting the solution of the above optimization problem by $^0 \bar q_t$, the value of being married for the spouse of gender $j$ at period $t$ is given by:

\begin{equation*}
^{M}V_t^j (\Omega_t) = u(^0 \bar c_t^j, ^0 \bar h_t^j, m_t = 1) + \beta E \left (^MV^j_{t+1}(\Omega_{t+1}) | \Omega_t, ^{0} \bar q_t \right ) 
\end{equation*}

If the couple decide to get divorce at period $t$, the family asset divides equally and the household will get the following value:

\begin{align*}
\ce{^{M1}V_t}(\Omega_t)  &= \max_{q_t} \quad  \theta^H \left ( u(c_t^H, h_t^H, m_t =0)  + \beta E(^{RD}V^j_{t+1} (\Omega_{t+1})| \Omega_t, q_t) \right) \\
  & + (1 - \theta^H) \left ( u(c_t^W, h_t^W, m_t = 0) + \beta E(^{RD}V^j_{t+1} (\Omega_{t+1})| \Omega_t, q_t) \right )\\
s.t. & \quad   c_t^j e(n_t) + A_{t+1}^j = (1+r) A_t^j + w_t^j h_t^j (1 - \tau) \quad j \in \{H, W\} \\
& A_t ^j = \dfrac{A_t}{2}
\end{align*}
 
\subsection{Solving The Model}

Starting backward, we obtain the following value functions for all values of state-space at each period:

\begin{equation*}
\Psi_t =\{ ^{R}V_t, ^DV^j_t, ^{M1}V_t, ^{M0}V_t\}
\end{equation*}

\noindent Calculating all of these value functions there are two continuous variable, $\{c_t^H, c_t^W\}$, and two discrete variables $\{h_t^H, h_t^W\}$. Calculating $^RV_t^j, ^{M1}V_t^j$ requires an unconstrained optimization problem and  calculating $^{M0}V_t$ involves a constrained optimization with two inequality constraints. Calculating $^DV_t^j$ involves two separate optimization problem to determine $\{c_t^H, h_t^H\}$ and $\{c_t^W, h_t^W\}$.  Solving these optimization problems paves the way for calculating the subsidiary value functions:

\begin{equation*}
\Psi_t'  =\{ ^{R}V_t^j, ^MV^j_t, ^{M}V_t, ^{RD}V_t^j\}
\end{equation*}

\noindent The terminal conditions for state variables are given by the following: 

\begin{align*}
A_{T+1}^j &= 0  \quad j \in \{H, W\} \\ 
\end{align*}





% --------------------- Bibliography hidden with a save box ---------------------------------------
\clearpage
\bibliographystyle{chicago}
\bibliography{/Users/Mohsen/Dropbox/heckman.bib, /Users/Mohsen/Dropbox/Mohsen.bib }

\end{document}